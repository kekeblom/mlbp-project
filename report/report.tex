\documentclass[twoside]{article}

\usepackage{cite}
\usepackage{listings}

\usepackage[sc]{mathpazo} % Use the Palatino font
\usepackage[OT1]{fontenc} % Use 8-bit encoding that has 256 glyphs
\usepackage[utf8]{inputenc}
\linespread{1.05} % Line spacing - Palatino needs more space between lines
\usepackage{microtype} % Slightly tweak font spacing for aesthetics
\usepackage{graphicx}

\usepackage[hmarginratio=1:1,top=32mm,columnsep=20pt]{geometry} % Document margins
\usepackage{multicol} % Used for the two-column layout of the document
\usepackage[hang, small,labelfont=bf,up,textfont=it,up]{caption} % Custom captions under/above floats in tables or figures
\usepackage{booktabs} % Horizontal rules in tables
\usepackage{float} % Required for tables and figures in the multi-column environment - they need to be placed in specific locations with the [H] (e.g. \begin{table}[H])
\usepackage{hyperref} % For hyperlinks in the PDF

\usepackage{lettrine} % The lettrine is the first enlarged letter at the beginning of the text
\usepackage{paralist} % Used for the compactitem environment which makes bullet points with less space between them

\usepackage{abstract} % Allows abstract customization
\renewcommand{\abstractnamefont}{\normalfont\bfseries} % Set the "Abstract" text to bold
\renewcommand{\abstracttextfont}{\normalfont\small\itshape} % Set the abstract itself to small italic text

\usepackage{titlesec} % Allows customization of titles
\renewcommand\thesection{\Roman{section}} % Roman numerals for the sections
\renewcommand\thesubsection{\Roman{subsection}} % Roman numerals for subsections
\titleformat{\section}[block]{\large\scshape\centering}{\thesection.}{1em}{} % Change the look of the section titles
\titleformat{\subsection}[block]{\large}{\thesubsection.}{1em}{} % Change the look of the section titles

\usepackage{amsmath}

\title{\vspace{-15mm}\fontsize{24pt}{10pt}\selectfont\textbf{Comparing Random Forests and Feedforward Neural Networks in sentiment analysis}}

\author{
  \large
  \textsc{Kenneth Tor Blomqvist}\\[2mm]
  \tt kenneth.blomqvist@aalto.fi \\[2mm]
}
\date{}

%----------------------------------------------------------------------------------------

\begin{document}

\maketitle % Insert title

%----------------------------------------------------------------------------------------
%	ABSTRACT
%----------------------------------------------------------------------------------------

\begin{abstract}
  \noindent{
This paper compares random forests and feedforward neural networks for classification and regression. An experimental evaluation is done on a dataset of yelp reviews. A theoretical overview of the two methods are given. The implementation of a feedforward neural network is explained. The best performing method for both classification and regression was the feedforward neural network with a small margin.
}

%%% Local Variables:
%%% mode: latex
%%% TeX-master: "report"
%%% End:

\end{abstract}

%----------------------------------------------------------------------------------------
%	ARTICLE CONTENTS
%----------------------------------------------------------------------------------------

\begin{multicols}{2} % Two-column layout throughout the main article text

  \section{Introduction}
  \label{sec:introduction}
  \noindent{

Neural networks are all the rage with the kids these days. News articles are constantly published about how neural nets are being applied with staggering results to image recognition, speech recognition, image synthesis, pattern recognition and other problems.
}

This paper does not mean to provide an in-depth analysis of the best possible approach for a given method. Instead this papers seeks to compare the performance characteristics and ease of use of a simple feed-forward neural network implementation to random forests. We conduct experiments against a dataset containing yelp reviews for a classification and a regression task.

Our dataset consists of yelp reviews of which the count of 50 different words in the review are given without any contextual information. The regression task consists of predicting how many people found a review useful based on the labeled training data. The classification task consists of trying to predict wether the review received over 3 "stars" from other users of the service.

Experiments are conducted against these tasks to compare the performance characteristics and other properties of these two very different methods.

This paper is laid out in several sections. In section ~\ref{sec:methods} a general overview of the compared methods is given. Section ~\ref{sec:experiments} describes the experiments performed and section ~\ref{sec:results} presents the results of the experiments. Section ~\ref{sec:discussion} gives a short discussion of the results.

%%% Local Variables:
%%% mode: latex
%%% TeX-master: "report"
%%% End:



  \section{Methods}
  \label{sec:methods}
  
\subsection{Feedforward Neural Networks}
Feedforward neural networks a classic deep learning model. They are called neural networks because they are formed by composing together different functions multiplying the input by matrices, possibly adding a bias temr and appliying activation functions. Each layers output is then passed to the next layer. The activation functions can be thought of as "neurons" and the matrices and biases as connections between them.

Neural networks can be used to approximate functions. However, as opposed to linear models neural networks can not be trained by fitting them using a closed form optimization method. The way neural networks are trained is that we define some cost function as the optimization goal. We then minimize the cost function using gradient based learning. The backpropagation algorithm is used to update the weights and biases of the model. \cite{deep-learning-book}.

In this paper we examine perhaps the most simple form of a feedforward neural network: a network with only fully connected layers. Fully connected layers are layers where the input is multiplied by a weight matrix, a bias term is added and the result is passed into some activtion function. Typical activation functions include sigmoid, tanh and restricted linear unit functions.

Neural networks are very robust and work very well in high dimensional settings. They are however computationally expensive to train. They also have quite a lot of parameters that can be tuned. This makes finding the optimal parameters time consuming. Grid search or other similar techniques can however be used to help in the search \cite{deep-learning-book}.

\subsection{Random Forests}

Decision trees are a non parametric learning method. The hypothesis is represented using a binary tree where at each node we make a binary decision. For classification each leaf node represents the output class and for regression the labels for the training data are averaged to get the prediction. Decision trees are learned by recursively expanding the tree using one of many learning algorithms available. A good overview of learning decision trees is given in \cite{alpaydin}.

Random forests are an averaging based ensemble method specifically designed for decision trees. \cite{sklearn} Several models are trained by sampling with replacement from the training set. The predictions of each model is then averaged to construct the final predictions. This way we can use several individual models with high variance and use averaging to reduce the overall variance. Decision trees typically can have high variance suffering from over fitting. They thus lend themselves very well for ensemble methods.




  \section{Experiments}
  \label{sec:experiments}
  
In the experiments we studied two different tasks. Classifiying yelp reviews into two classes: those that received over 3 stars and those that didn't. In the regression task we try to predict how many people find the review useful.

The datasets both consist of 50 different features for each datapoint. Each feature indicates how many occurences of a word where found in the review text. The data consists of 5,000 training examples and a disjoint test set of 1,000 test examples.

\subsection{The Neural Network}

The neural network was trained by using stochastic gradient descent and the backpropagation algorithm to tune the weights and biases. The validation dataset is split off from the training data and it's size is 1/10th of all the available training data.

For optimization the RMSProp algorithm was implemented as presented in \cite{deep-learning-book}. Theano \cite{theano} was used to implement the computational graph and to derive the gradients.

For the regression task we used mean squared error as a cost function to minimize the error in our network. For the classification task categorical cross entropy was used.

The network was trained until the classification accuracy no longer improved on the validation set for more than 5 passes over the training dataset. The network was then evaluated on the test dataset and the test accuracy logged.

Since there is a vast amount of different hyperparameters available for tuning, the model was tuned by hand. The network parameters were adjusted after each run. Different networks depths and widths where tried along with tanh, sigmoid and rectified linear unit activation functions. Several l2 norm regularization weights were tried. Many different learning rates ranging from 0.1 to 0.0001 were tried.

The full implementation of the neural network can be found on github at \hyperref{https://github.com/kekeblom/mlbp-project}. The implementation of the RMSProp algorithm is included in appendice A \ref{appendix-a}.

\subsection{The Random Forests}


For both tasks the models were tuned by running a grid search to find the best possible model parameters. The parameters that where tuned were the minimum samples left at a leaf of a tree and the maximum depth of each individual tree. The values for the minimum samples at a leaf node ranged from 1 to 10 with a step of 1. The values for the maximum depth ranged from 1 to 48 with a step of 3.

In our experiments we used the scikit-learn random forest implementation \cite{sklearn}. The k-fold cross-validation implementation was also from the scikit-learn library.

At each step of the search, the model was evaluated using k-fold cross validation with 5 folds. The random forest that achieved the best k-fold validation accuracy was used to evaluate the model on the test dataset.




  \section{Results}
  \label{sec:results}
  I tried diferent network depths from 1 to 4 layers deep along with different layer widths. I also tried a number of different activation functions for all of the layers namely tanh, sigmoid and rectified linear units.

In the initial experiments the models tended to plateau around a final test accuracy of about 62\%. I tried to adjust all of the parameters including learning rate but the models would get stuck in that plateau. This got me thinking the model migth be suffering from overfitting and therefore might benefit from some regularization scheme. I added the weighted l2 norm of the weights as a regularization term to the cost function. This substantially improved the final test set accuracy. After trying out different weights I settled for a weight of 0.5.

The network that turned out to perform the best was a two layer network


  \section{Discussion}
  \label{sec:discussion}
  
We found that random forests and neural networks achieve similar performance on a small, relatively high dimensional dataset such as the one used in this paper. The neural network performed slightly better on both tasks. The neural network turned out to perform much better on both tasks.

Despite the fact that the neural network achieved good results on both tasks, random forests might still be a better option in many cases. Random forests are much easier to use and a well perfoming model can be found very quickly. Random forests are computationally faster to train which means that more variations can be tried quicker.

Different options for the count of the random forest estimators in the models were not tried. This could be an interesting experiment to run to see wether better results could be achieved by adding more estimators. Other parameters could also have been included in the search, such as the maximum number of samples at a leaf node or different node splitting criteria. Some feature selection scheme could have been tried as decision trees are known to not perform very well in high dimensional settings as they use a heuristic search for growing the tree to fit the dataset.

Better results and less biased results could have been achieved using a grid search for finding the neural network parameters, but since there were so many parameters to be tuned and a lack of time, the decision was made to hand tune the parameters.





  %	REFERENCE LIST
  %----------------------------------------------------------------------------------------

  \bibliographystyle{IEEEtran}
  \bibliography{references}

\end{multicols} % One-column layout for the appendices
\pagebreak

\section{Appendix A: Implementation of the RMSProp algorithm}
\label{appendix-a}

This appendice contains the code for the RMSProp algorithm. The full code can be found at \url{https://github.com/kekeblom/mlbp-project}.

\lstinputlisting[language=Python, firstline=41, lastline=56, basicstyle=\tiny]{../neural_net/optimizers.py}




\section{Appendix B: Implementation of stochastic gradient descent}
\label{appendix-b}

This appendice contains part of the code for running stochastic gradient descent for the neural network.

\lstinputlisting[language=Python, firstline=76, lastline=99, basicstyle=\tiny]{../neural_net/model.py}


\section{Appendix C: Implementation of the fully connected layer}
\label{appendix-c}

This appendice contains the code for the fully connected neural network layer implementation.

\lstinputlisting[language=Python, basicstyle=\tiny]{../neural_net/fullyconnected.py}


%----------------------------------------------------------------------------------------

\end{document}

%%% Local Variables:
%%% mode: latex
%%% TeX-master: t
%%% End:
